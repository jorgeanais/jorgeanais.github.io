%------------------------------------
% Dario Taraborelli
% Typesetting your academic CV in LaTeX
%
% URL: http://nitens.org/taraborelli/cvtex
% DISCLAIMER: This template is provided for free and without any guarantee 
% that it will correctly compile on your system if you have a non-standard  
% configuration.
% Some rights reserved: http://creativecommons.org/licenses/by-sa/3.0/
%------------------------------------


%!TEX TS-program = xelatex
%!TEX encoding = UTF-8 Unicode

\documentclass[11pt, a4paper]{article}
\usepackage{fontspec} 



% DOCUMENT LAYOUT
\usepackage{geometry} 
\geometry{a4paper, textheight=25cm, marginparsep=-2.8cm, marginparwidth=2.8cm, left=2cm, right=2cm}
\setlength\parindent{0in}

% FONTS
\usepackage{xunicode}
\usepackage{xltxtra}
\defaultfontfeatures{Mapping=tex-text} % converts LaTeX specials (``quotes'' --- dashes etc.) to unicode
\setromanfont [Ligatures={Common},Numbers={OldStyle}]{Adobe Caslon Pro}
\setmonofont[Scale=0.8]{Monaco} 
\setsansfont [Ligatures={Common}, BoldFont={Fontin Sans Bold}, ItalicFont={Fontin Sans Italic}]{Fontin Sans}
% ---- CUSTOM AMPERSAND
\newcommand{\amper}{{\fontspec[Scale=.95]{Adobe Caslon Pro}\selectfont\itshape\&}}
% ---- MARGIN YEARS
\usepackage{marginnote}
\newcommand{\years}[1]{\marginnote{ #1}}
% \renewcommand*{\raggedrightmarginnote}{}
% \renewcommand*{\raggedleftmarginnote}{}
% \reversemarginpar

% HEADINGS
\usepackage{sectsty} 
\usepackage[normalem]{ulem} 
\sectionfont{\rmfamily\mdseries\Large} 
\subsectionfont{\rmfamily\mdseries\scshape\normalsize} 
\subsubsectionfont{\rmfamily\bfseries\upshape\normalsize} 

%Loading xcolor with dvipsnames
\usepackage[dvipsnames]{xcolor}

% PDF SETUP
% ---- FILL IN HERE THE DOC TITLE AND AUTHOR
\usepackage[xetex, bookmarks, colorlinks, breaklinks, pdftitle={Jorge Anais - vita},pdfauthor={Jorge Anais}]{hyperref}  
\hypersetup{linkcolor=teal,citecolor=teal,filecolor=black,urlcolor=teal} 

\usepackage{multicol}

% DOCUMENT
\begin{document}
{\LARGE Jorge Anais Vilchez} \\

\hrule

\vspace{0.3cm}


\section*{Education}
\years{Expected 2020} \textbf{Universidad de Antofagasta}, Chile \\
\textit{Master's candidate in Astronomy}\\
% {\footnotesize  Thesis: Search and Characterization of Star Cluster Candidates at the Far End of the Galactic Bar}\\
{\small Advisors: Dr. Sebastian Ramírez Alegría \amper{} Dra. Karla Peña Ramírez}\\

\years{2014} \textbf{Potificia Universidad Católica de Chile}\\
\textit{Licenciate in Astronomy}\\
{\small Advisor: Dra. Manuela Zoccali}\\

\section*{Further Degrees}
\years{2018} \textbf{Duoc UC}, Santiago, Chile.\\
\textit{Diploma in Software Development}\\

\years{2016} \textbf{Universidad de Antofagasta}, Antofagasta, Chile.\\
\textit{Diploma in Astro-engineering}\\


\section*{Work \amper{} Research Experience}

\years{2020-present} \textbf{Research Assistant}, CITEVA, Universidad de Antofagasta.\\
\textit{Study on transmission spectroscopy models for Earth-like and Mini-Neptunes \\planetary atmospheres.}\\
{\small Under the supervision of Dr. Jeremy Tregloan-Reed}\\

\years{2017-present} \textbf{Observer}. Las Campanas Observatory.\\
\textit{1M2H survey. Photometric transients follow-up using Swope Telescope.}\\
{\small Under the supervision of Dr. Ryan Foley}\\

\years{2016} \textbf{Research Assistant}, IA, Universidad Católica del Norte.\\
\textit{Data reduction FORS and Mage instruments.}\\
{\small Under the supervision of Dr. Christian Moni-Bidin}\\

\years{2014-2015} \textbf{Telescope Operator and Technical Assistant}, LCO.\\
\textit{CSP II. Photometric SN follow-up using Swope Telescope.}\\
{\small Under the supervision of Dr. Mark M. Phillips}\\

\years{2014} \textbf{Research Assistant.} Santa Martina Observatory, PUC.\\
\textit{Data Reduction and observations using ESO50/PUCHEROS spectrograph.}\\
{\small Under the supervision of Dr. Leonardo Vanzi.}\\ 

\years{2012-2013} \textbf{Research Assistant.} IMUC, PUC.\\
\textit{Research and Software develpment for medical imaging sonification.}\\
{\small  Under supervision of Dr. Rodrigo Cádiz and Dr. Patricio de la Cuadra.}\\

\years{2011} \textbf{Summer internship}. IA, PUC.\\
\textit{Commissioning of the ESO50 telescope. }\\
{\small Under supervision of Dr. Leonardo Vanzi.}


\section*{Observing Experience}

\textbf{Las Campanas Observatory}.\\ Swope 1m Telescope / CCD (>250 nights)\\

\textbf{UA Ckoirama Observatory}.\\ Chakana 0.6 m telescope / CCD\\

\textbf{Calar Alto Astronomical Observatory}.\\ Zeiss 1.23m telescope / CCD\\

\textbf{PUC Santa Martina Observatory}.\\ ESO 50cm Telescope / PUCHEROS Spectrograph



\section*{Teaching experience}
\years{2010-2014} \textbf{Pontificia Universidad de Chile.}\\
\textit{Teaching assistant} for the following courses: Introduction to physics for biological\\
 sciences, Static and Dynamics for Civil Engineers, Thermodynamics, Optics,\\
 Astronomical Instrumentation and Electricity and Magnetism.


\section*{Software Skills}
\textbf{Astronomical}: IRAF, Topcat.\\
\textbf{Computer Programming}:  Python, Java, C, SQL, UNIX Shell scripting, MATLAB.\\
\textbf{Productivity Applications}: \TeX \ (\LaTeX, BibTeX) and most common office software.\\
\textbf{Operating Systems}: Linux, Windows.
\textbf{Github:} \texttt{https://github.com/jorgeanais}

\section*{Languages}

Spanish (native) and English 


\section*{Outreach Activities}
\years{2019} Public Talk: Formación de Sistemas Planetarios y Panspermia. UA.\\
\years{2018} Facilitador Museo Interactivo Mirador MIM.\\
\years{2016} Encargado difusión. Departamento de Física, UCN.\\
\years{2010-2012} Member of \href{https://fisicaitinerante.cl/}{Física Itinerante}.\\

\section*{Conferences \amper{} Schools}

Antofagasta, Chile, 3-8 November 2019. \textit{The XVI Latin American Regional IAU Meeting.}\\

York, United Kingdom, 16-20 September 2019. \textit{Conference From Gas to Stars: The Links between Massive Star and Star Cluster.} \textbf{Contributed Poster}: Massive Open Clusters in VVV data using unsupervised clustering algorithms.\\
 
La Serena, Chile, August 19-28, 2019. \textit{La Serena School for Data Science. Applied Tools for Data Driven Science}. AURA Observatory.\\

% Santiago, Chile, 27-29 January 2016. \textit{Workshop: Programming with CUDA} \\

\section*{Publication List}
\begin{enumerate}
  \item Tregloan-Reed, J., Otarola, A., Ortiz, E., Molina, V., \textbf{Anais, J.}, González, R., Colque, J. P., Unda-Sanzana, E., 2020, A\&A, 637, L1. First observations and magnitude measurement of Starlink's Darksat.
  \item Burns, C. R., Ashall, C., Contreras, C.  et al. including \textbf{Anais, J.} 2020, submitted to ApJ, arXiv:2004.13069. SN 2013aa and SN 2017cbv: Two Sibling Type Ia Supernovae in the spiral galaxy NGC 5643.
  \item Stritzinger, M. D., Taddia, F., Fraser, M., et al. including \textbf{Anais, J.} 2020, submitted to A\&A, arXiv:2005.00319.  The Carnegie Supernova Project II. Observations of the intermediate luminosity red transient SNhunt120.
  \item Stritzinger, M. D., Taddia, F., Fraser, M., et al. including \textbf{Anais, J.} 2020, submitted to A\&A, arXiv:2005.00076. The Carnegie Supernova Project II. Observations of the luminous red nova AT 2014ej.
  \item Holmbo, S.; Stritzinger, M. D.; Shappee, B. J., et al. including \textbf{Anais, J.} 2019, A\&A, 627, A174. Discovery and progenitor constraints on the Type Ia supernova 2013gy.
  \item Phillips, M. M., Contreras, C., Hsiao, E. Y., et al. including \textbf{Anais, J.} 2019, PASP, 131, 014001. Carnegie Supernova Project-II: Extending the Near-infrared Hubble Diagram for Type Ia Supernovae to z$\sim$0.1.
  \item Burns, Christopher R.; Parent, Emilie; Phillips, M. M., et al. including \textbf{Anais, J.} 2018, ApJ, 869, 56. The Carnegie Supernova Project: Absolute Calibration and the Hubble Constant.
  \item Stritzinger, M. D., Anderson, J. P., Contreras, C., et al. including \textbf{Anais, J.} 2018, A\&A, 609, A134. The Carnegie Supernova Project I. Photometry data release of low-redshift stripped-envelope supernovae.
  \item Krisciunas, K., Contreras, C., Burns, C. R., et al. including \textbf{Anais, J.} 2017, AJ, 154, 211. The Carnegie Supernova Project. I. Third Photometry Data Release of Low-redshift Type Ia Supernovae and Other White Dwarf Explosions.
  \item Coulter, D. A., Kilpatrick, C. D., Foley, R. J., \textbf{Anais, J.} et al. 2017, ATel, \#10167: Swope Photometric Observations of SN 2017cbv$=$ DLT17u.
\end{enumerate}

\section*{References}
References available upon request

\vspace{1cm}
\begin{center}
{\small Last updated \today\- •\- Santiago, Chile}\\
{\scriptsize  Personal webpage curriculum vit\ae.}
\end{center}

\end{document}
